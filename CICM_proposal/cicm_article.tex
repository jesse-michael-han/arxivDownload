\documentclass[a4paper]{easychair}

\usepackage{listings}
\usepackage{hyperref}
\usepackage{listings}
%\usepackage{tabularx}

\usepackage{color}
\definecolor{gray}{rgb}{0.4,0.4,0.4}
\definecolor{darkblue}{rgb}{0.0,0.0,0.6}
\definecolor{cyan}{rgb}{0.0,0.6,0.6}

\lstset{
  basicstyle=\ttfamily,
  columns=fullflexible,
  showstringspaces=false,
  commentstyle=\color{gray}\upshape
}

\lstdefinelanguage{XML}
{
  morestring=[b]",
  morestring=[s]{>}{<},
  morecomment=[s]{<?}{?>},
  stringstyle=\color{black},
  identifierstyle=\color{darkblue},
  keywordstyle=\color{cyan},
  morekeywords={xmlns,version,type}% list your attributes here
}

\authorrunning{Berlioz, Luis}
\titlerunning{Creating a Database of Definitions From Large Mathematical Corpora}
\author{Luis Berlioz}
\title{Project Proposal: Creating a Database of Definitions From Large Mathematical Corpora}
\institute{
    University of Pittsburgh\\
    \email{lab232@pitt.edu}
}
\begin{document}
\maketitle
\begin{abstract}
   We propose a method to gather large amounts of  definitions from  mathematical documents available online.
Recent work indicates that text classification algorithms can have excellent accuracy  at determining when a certain paragraph is in fact a definition or not. These algorithms are trained on large math corpora available online like the arXiv website. The \LaTeX{}    source code of these documents is first converted into a more structured format like XML or HTML with the software package LaTeXML. The training data for the classifier is then obtained by searching for the definitions that the author labeled with a \LaTeX{} macro. The second phase of the system consists of extracting the term being defined from each definition. This task is performed by a Named Entity Recognition (NER) model trained using data from websites with mathematical content. The data is finally organized according to several different properties like semantic similarity and content dependency. 

\end{abstract}
\section{Introduction}
In this paper we describe a system for the extraction of definitions and definienda from large collections of digital mathematical documents like the arXiv website. The main objective of this system is to organize all the mathematical lexicon both by dependency and semantically.  We also go over the implementation of a prototype of such system. As well as the processing of the different sources of digital documents used to create this first implementation. The resulting system, although unfinished, provides a convincing proof  of concept as well as a baseline for the development of more effective systems of this type in the future.

\section{Obtaining the Data}
The two main sources of data used in this project are the arXiv and Wikipedia websites. To download the data from the arXiv without affecting the website's traffic, we used the bulk download service\footnote{https://arxiv.org/help/bulk\_data\_s3}. The \LaTeX{} source of each article is compressed together in large tar files. Similarly, the Wikipedia data can be downloaded as a compressed multistream file\footnote{https://dumps.wikimedia.org/enwiki/}. 

\subsection{Processing the arXiv articles}
The \LaTeX{} source from the arXiv has to be further processed before it becomes useful. This is why the \LaTeX{} source was converted to a more structured format using the software package LaTeXML \cite{miller3latexml}. LaTeXML converts the \TeX{} source first to xml and optionally to html by using an additionally script. 

\lstset{language=XML,
basicstyle={\scriptsize\ttfamily},}
\begin{center}
\begin{figure}[h]
\begin{lstlisting}
    <theorem class="ltx_theorem_definition" inlist="thm theorem:definition" xml:id="Thmdefinition1">
      <tags>
        <tag>Definition 1</tag>
        <tag role="refnum">1</tag>
        <tag role="typerefnum">Definition 1</tag>
      </tags>
      <title class="ltx_runin"><tag><text font="bold">Definition 1</text></tag>.</title>
      <para xml:id="Thmdefinition1.p1">
        <p class="ltx_emph"><text font="italic">Let <Math mode="inline" 
                             tex="k" text="k" xml:id="Thmdefinition1.p1.m1">
              <XMath>
                <XMTok role="UNKNOWN">k</XMTok>
              </XMath>
\end{lstlisting}
    \caption{\label{xml1} example of LaTeXML output of a definition in an article}
\end{figure}
\end{center}

%\begin{center}
%    \begin{figure}[h]
%\begin{tabular}{ll}
%    \LaTeX{}  Source & XML Output \\
%    \verb|nice to meet you\n|
%    \verb|this is so nice|
%     &
%    all you can eat 
%\end{tabular}
%    \end{figure}
%\end{center}

\section{Classifying Definitions}
Recent work indicates that well known text classification algorithms \cite{bengio2003neural,chen2017improving} can have excellent accuracy  at determining whether certain paragraph is in fact a definition \cite{webscipara}. The content of the arXiv articles are tokenized and fed into a word embedding algorithm like GloVe \cite{pennington2014glove}. This has been  implemented already and is available in \cite{SML}. 

As training data for the classifier, we use the passages of certain articles that are labeled as definitions by the author by placing them in certain \LaTeX{} macro environments. These macros are normally defined in the preamble of the document using the \textbackslash \texttt{newtheorem} macro. LaTeXML deals with the user defined macros and tags the corresponding text in the output like in figure \ref{xml1}. We have performed successful experiments  using common general purpose algorithms implemented in the scikit--learn Python library \cite{scikit-learn}. And these were confirmed with the results shown on the website \url{https://corpora.mathweb.org/classify_paragraph}.
In table \ref{sanity} we can observe the result of the classifier on some simple examples.

\begin{table}[h]
    \begin{center}
    \begin{tabular}{|p{0.5\textwidth}|c|}
        \hline
        \hline
        \textbf{Input to the Classifier} & \textbf{Result} \\
        \hline
        \hline
        a banach space is defined as a complete vector space. & \textbf{True}\\
        \hline
        This is not a definition honestly. even if it includes technical words like scheme and cohomology & \textbf{False} \\
        \hline
        There is no real reason as to why this classifier is so good. & \textbf{False}\\
        \hline
        a triangle is equilateral if and only if all its sides are the same length. & \textbf{True}\\
        \hline
    \end{tabular}
    \caption{\label{sanity}simple examples of the behaviour of the classifier}
    \end{center} 
\end{table}

 Text classifiers normally take each paragraph of an article and output an estimate of the probability of it being a definition or not.  Figure \ref{showdown} present the basic performance metrics of the some of the classifiers implemented in the scikit-learn library. The Support Vector Classifier was observed to have the best performance and a more detailed view of the result is pictured in table \ref{metrics}. In the future we plan  to use the \textit{fasttext} method in \cite{bagof} which has the best tradeoff between classification speed and accuracy.  

\begin{figure}
    \centering
    \begin{minipage}{0.45\textwidth}
        \centering
        {\small \begin{verbatim}
==============================
MultinomialNB ,  ngrams=(1,4)
****Results****
Accuracy: 86.6386%
Log Loss: 2.723683288081348
==============================
MultinomialNB
****Results****
Accuracy: 86.1733%
Log Loss: 1.8957941996159562
==============================
SVC ,  C= 2000
****Results****
Accuracy: 89.3283%
Log Loss: 0.29110830190582887
==============================
NuSVC
****Results****
Accuracy: 84.0215%
Log Loss: 0.34342025343628446
        \end{verbatim}}

    \end{minipage}\hfill
    \begin{minipage}{0.45\textwidth}
        \centering
        {\small
        \begin{verbatim}
==============================
DecisionTreeClassifier
****Results****
Accuracy: 81.3609%
Log Loss: 6.437730639470123
==============================
RandomForestClassifier
****Results****
Accuracy: 83.6580%
Log Loss: 0.6044826387423514
==============================
AdaBoostClassifier
****Results****
Accuracy: 84.4868%
Log Loss: 0.6717126298133219
==============================
GradientBoostingClassifier
****Results****
Accuracy: 85.8098%
Log Loss: 0.3531810109520398
        \end{verbatim}}
    \end{minipage}
        \caption{\label{showdown}comparison of the most common classification algoriths on classifying definitions}
\end{figure}

\begin{table}[h]
    \begin{center}
    \begin{tabular}{|c|c|c|c|c|}
        \hline
          & precision  &  recall &  $F_1$-score & support\\
\hline
         nondefs  &   0.73   &  0.91  &   0.81   &    2,217\\
         \hline
         definitions   &    0.95   &   0.84  &    0.89  & 4,661\\
         \hline
         \multicolumn{5}{}{}\\
         \hline
   micro avg   &    0.86  &    0.86  &    0.86   &   6,878\\
         \hline
   macro avg   &    0.84  &    0.87  &    0.85   &   6,878\\
         \hline
weighted avg   &    0.88  &    0.86  &    0.87   &   6,878\\
         \hline
    \end{tabular}
        \caption{\label{metrics} Overall performance of the SVC classifier on the test set}
    \end{center} 
\end{table}


\section{Extracting Definienda}
After determining the definitions in the text, the system is required to find what is the term that is being defined in each definition. It is assumed that the  \emph{definiendum} is one or more adjacent words in the definition. This task can be interpreted as a Named Entity Recognition (NER) problem. Several different tecniques have been developed to deal with it;  as it is one of the most important subtasks of Information Extraction \cite{nersurvey}.

 For the first approach to this problem, we used the ChunkParserI package from the NLTK library \cite{nltk}. This module uses a supervised learning algorithm that is trained on examples of definitions tagged with  part of speech and IOB which is this case is the inside, outside and beginning of the definiendum (see table \ref{iobtags}). After the training  is done, the model tries to predict the IOB tag with only the first two rows as input.

To obtain the tagged text, the whole body of text from Wikipedia was used. The examples of definitions were obtained by filtering the articles with the two following properties:
\begin{itemize}
    \item Articles that have a section with the word \textit{definition}.
    \item The title of the article must appear at least once in this section.
\end{itemize}
These sections were assumed to be definitions and the title of the article which they belong to the definuendum. 

Several difficulties were observed with this approach. 
\begin{table}[h]
    \begin{center}
        {\scriptsize
        \begin{tabular}{lcccccccccc}
        \hline
            \textbf{text} &We & define & a & Banach & space & as & a & complete & vector & space \\
            \textbf{POS}&PRP & VBP & DT & NNP & NN & IN & DT & JJ & NN & NN\\
            \textbf{IOB} & O& O& O& B--DFNDUM & I--DFNDUM & O& O& O& O& O \\
         \hline
    \end{tabular}
        \caption{\label{iobtags} Input example used in  training the IOB parser.}}
    \end{center} 
\end{table}
 
\bibliographystyle{plain}
\bibliography{cicm_article}
\end{document}
